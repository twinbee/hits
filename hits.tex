%% last modified April 3, 2006 by M bennett to meet IEEE specifications

\documentclass[times, 10pt,twocolumn]{article}
\usepackage{ieee8}
\usepackage{times}

\usepackage{graphics}                 % Packages to allow inclusion of graphics
\usepackage{graphicx}
\usepackage{hyperref}                 % For creating hyperlinks in cross references
\usepackage{listings}                 % Source code listing

%\topmargin -1.5cm \oddsidemargin -0.04cm \evensidemargin -0.04cm
%\textwidth 16.00cm \textheight 23.50cm
%\parskip 7.2pt \parindent 0.25in
%\makeindex

\title{A Scalable Parallel HITS Algorithm for Page Ranking }

\author{Matthew Bennett, Julie Stone, Chaoyang Zhang \\
{{\em School of Computing. University of Southern Mississippi.
Hattiesburg, MS 39406 } } \\ {\em matthew.bennett@usm.edu,
julie.stone@usm.edu, chaoyang.zhang@usm.edu }  }

 \date{ }

\begin{document}
\pagestyle{empty} \maketitle \thispagestyle{empty}
\begin{abstract}
The Hypertext Induced Topic Search (HITS) algorithm is a method of
ranking authority of information sources in a hyperlinked
environment. HITS uses only topological properties of the
hyperlinked network to determine rankings. We present an efficient
and scalable implementation of the HITS algorithm that uses MPI as
an underlying means of communication. We then analyze the
performance on a shared memory supercomputer, and use our results to
verify the optimal number of processors needed to rank a large
number of pages for the link structure of the total University of
Southern Mississippi (usm.edu domain) web sites. \end{abstract}

%\noindent Keywords: parallel algorithms, page ranking, information
%retrieval, graph algorithms, world wide web, speedup, efficiency

\Section{Introduction}
 % \addcontentsline{toc}{Section}{Introduction}

Page ranking is an important part of information retrieval, related
to searching through a group of web pages or other similar cross
referenced resources. For example, when an internet search engine is
used, a ranking is used to sort the resulting pages in terms of
relevance and authority. Obviously, there is no absolute measure of
the authority of a given resource, but approximations do exist
\cite{kleinberg}. It is important to have a constantly up-to-date
compendium of page rankings, so that rankings relevant to a given
search are no more than a few days old. Otherwise, the ranking is
useless, since internet resources change so quickly, constantly
replaced with other authoritative content.

This is problematic, since page ranking algorithms such as HITS
\cite{kleinberg} and Google's PageRank \cite{google} are by their
nature computationally expensive, and the number of accessible web
pages which must be considered is incredible. One author estimated
in January 2005 the total indexable number of web pages to be 11.5
billion \cite{gulli}. Since internet growth has been explosively
exponential, the total number of pages a major web search engine
such as Google should index and rank will be tremendous. Page
ranking in information retrieval is a good candidate for massively
parallel systems that can constantly compute and cache large amounts
of meta-data associated with information retrieval tasks.

The hypertext induced topic search (HITS) algorithm, developed by
Kleinberg \cite{kleinberg}, utilizes the native link structures of a
collection of web pages to determine the authority and connectivity
of each. HITS assigns two attributes to a set of hypertext pages:
Hubs and Authorities. A Hub value is a measure of the number of
authoritative pages to which a page links. Each page has associated
with it a real value $H \epsilon [0,1]$, which gives an indication
of the quality and amount of links that page contains. An Authority
value for a given page p is a measure of the quantity and Hub values
of the pages that link to p. Each page has associated with it a real
value $A \epsilon [0,1]$, which gives an indication of the quality
and amount of links pointing to that page. Equations \ref{Authority}
can be used to calculate Authority values A and Hub values H,
respectively.

\begin{equation} \label{Authority} A^{p}(v) = \sum_{(k,v)} H^{p-1}(k)\ \ \ H^{p}(v) = \sum_{(v,k)} A^{p-1}(k)\end{equation}

The A value at time step $p$ for vertex $v$ is given by the sum of
the H values at time step $p-1$ of all pages which link to $v$.
Similarly, the H value at time step $p$ for vertex $v$ is given by
the sum of the A values at time step $p-1$ of all pages to which $v$
links. Kleinberg suggests it is usually sufficient to terminate the
algorithm and give rankings whenever $p \approx 20$. The algorithm
is said to converge whenever the rankings at iteration $k$ are
identical to the rankings at iteration $k-1$ \cite{kleinberg}. We
use a graph to describe the link structure between pages.
Internally, the link structure is represented as a standard
unweighted adjacency matrix, in computer science texts such as
\cite{corman}. \Section{Serial Algorithm}
 % \addcontentsline{toc}{Section}{Serial Algorithm}
  \newcounter{Lcount}
  \begin{list}{\Roman{Lcount}}
    {\usecounter{Lcount}
    \setlength{\rightmargin}{\leftmargin}}
 \item Create Adjacency Matrix M. If there is a hyperlink from page A to page B, then the cell at $M_{AB}$ is set to \texttt{true}.
 \item Calculate Hubs and Authorities. The initial values of hubs and authorities are the number of links from or to that node, respectively.
 \item Normalize Results so that the results gained map to $[0,1]$.
 \item Repeat II-III until the rankings stabilize.
 \item Sort Ranks. Sort out the ranks of the authorities to determine which page in the resultant set is the most relevant to this query.
 \end{list}

Normalization of a set of either Hub or Authority values is done in
the following way: Calculate the sum of the squares over all Hub or
Authority values, then divide the square of each value by the sum of
the squares. This may be done on a single machine in $\Theta(n)$
time, where $n$ is the number of pages considered. For instance, the
hub normalizations can each be calculated using equation
\ref{hubnorm} as suggested in \cite{kleinberg}.
\begin{equation}\label{hubnorm} h'=\frac{h}{\sqrt{\sum_{k \epsilon H}k^{2}}}\ \ \ a'=\frac{a}{\sqrt{\sum_{k \epsilon a}k^{2}}} \end{equation}

\Section{Parallel Design}
 % \addcontentsline{toc}{Section}{Parallel Design}

To calculate the authority value A$_{i} ^{t}$ for any given vertex
$i$ in our graph at time $t$, any implementation of the HITS
algorithm needs to know all of the previous k hub values $H_{k}
^{t-1}$ for those vertices that link to vertex $i$. Similarly, to
calculate the hub value H$_{i} ^{t}$ for vertex $i$ at time $t$,
HITS must know the previous authority values $A_{k} ^{t-1}$ for all
k vertices which $i$ links to.

If a processing element $p$ is responsible for calculating all
$H_{i}$ and $A_{i}$ for vertex $i$, then $p$ needs to know all
previous hub values for all vertices, and all previous authority
values for all vertices, and also the network topology in the form
of an adjacency matrix.

When the program begins execution, the adjacency matrix is read from
disk at process 0. The input file can be in one of three file types:
binary adjacency matrix, text adjacency matrix, or text adjacency
list. The graph's internal representation is that of an adjacency
matrix of Boolean values, which is flattened and broadcast to all
processes. All processors receive the adjacency matrix along with
its size. Each processing element is then delegated a number of
vertices.

The number of vertices delegated to each process, known as $step$,
is determined so that given $n$ vertices and $p$ processing
elements, the first $p\ (n/p - \lfloor n/p\rfloor)$ of the processes
are delegated $step=\lfloor n/p\rfloor$ vertices, and $p\ (1.0 -
(n/p - \lfloor n/p\rfloor))$ are delegated $step+1$ vertices. This
process assures that the nodes are divided equally among processes.
To assure that the division always works, the last process is always
assigned the last vertex as a last element.

For the initial iteration of the HITS algorithm, no communication
between nodes is necessary. The initial authority and hub sets can
be calculated using only information available in the adjacency
matrix. Each processing element calculates its designated hub values
as row sums of the adjacency matrix, and calculates its designated
authority values as column sums of the adjacency matrix. The
processes then perform a Gather to process 0 in such a way that each
process sends back only the Authority values for its designated set
of vertices. An identical Gather occurs on the Hub values.

At this point, the master process may do one of two things. If the
total number of iterations have not yet reached the desired number
of iterations, the master process normalizes the sets of Hub and
Authority values it has just gathered. It then broadcasts the Hub
and Authority sets to all processes, so that they may begin the next
iteration of the HITS algorithm. Otherwise, the master process sorts
the vertexes into two new lists, by Authority weight and by Hub
weight. These lists are the page rankings.

%\begin{figure}[hp]
%  \begin{center}
%    \includegraphics[scale=0.3]{images/sweetgum229.jpg}
%    \caption{\label{bbtime_sweetgum} Wall Time vs. processors for Block-and-broadcast on Sweetgum, input size 229}
%  \end{center}
%\end{figure}
%\begin{figure}[hp]
%  \begin{center}
%    \includegraphics[scale=0.3]{images/mimosa229.jpg}
%    \caption{\label{bbtime_mimosa} Wall time vs. processors for Block-and-broadcast on Mimosa, input size 229}
%  \end{center}
%\end{figure}
\begin{figure}[h]
  \begin{center}
    \includegraphics[scale=0.2]{images/sweetgum229speedup.jpg}
  \end{center}
 \caption{\label{bbspeed_sweetgum} Speedup vs. processors for Block-and-broadcast on Sweetgum, input size 229}
\end{figure}
\begin{figure}[h]
  \begin{center}
    \includegraphics[scale=0.2]{images/mimosa229speedup.jpg}
  \end{center}
 \caption{\label{bbspeed_mimosa} Speedup vs. processors for Block-and-broadcast on Mimosa, input size 229}
\end{figure}

Figures in \ref{bbspeed_sweetgum} show the speedup obtained using
the naive block-and-broadcast approach. (An explanation of speedup
is given on page \pageref{speedup}.) It is apparent that this
approach is not scalable, as the speedup actually decreases when the
number of processors is increased and the input size remains the
same. There are two reasons for this: the idle time introduced by
blocking, and also the communication cost associated with the
non-buffered broadcasts at each iteration. All processors must wait
to receive Hub and Authority broadcasts from a master single process
before proceeding with the next iteration of HITS. The idle
(blocked) time and communication cost of this approach imply a
better communication pattern.

An alternative approach is to replace the \{Gather to master
process, normalize, Broadcast\} method with the all-to-all Gather
facility supported by MPI \cite{gropp} and other message passing
libraries. In this approach, each process is responsible for
normalizing Hub and Authority values locally after each gather, and
no special broadcast is needed from the first process. This saves
significant communication and blocking time on massive problem
sizes, and can be seen in the pseudo code. Communication is
buffered, and occurs between all processes at the end of each
iteration. Though the hub and authority vectors must be calculated
locally, this can be done in substantially less time than the
blocking time required for the previous method.

\SubSection{Parallel Algorithm}
 % \addcontentsline{toc}{Section}{Parallel Algorithm}

%\lstset{commentstyle=\textit}
%\lstset{linewidth=\textwidth}\label{code:algorithm}
%\begin{lstlisting}[frame=trbl]{}
\noindent\textbf{Driver}
\begin{verbatim}
 iterations := 20
 curiter    := 0

 //p0 load the file, then
 //broadcast adj matrix
 if rank := 0
    load file
    if size > 1
     Broadcast Adj Matrix

 Split up the task evenly
  // (-> myFirst, myLast)

 while next(curiter) < iterations
  HITS(adj matrix, n,
        myFirst, myLast, a, h)
  if curiter > 1
   normalize authority values
   normalize hub values
  All-Gather Authorities and Hubs

 if (rank = 0)
    output the results
\end{verbatim}
%\end{lstlisting}

%\subSection{Modified HITS algorithm}
% % \addcontentsline{toc}{Section}{Modified HITS algorithm}
%\lstset{commentstyle=\textit} \lstset{linewidth=\textwidth}
%\begin{lstlisting}[frame=trbl]{}
\noindent\textbf{HITS (g, n, first, last, a, h)}
\begin{verbatim}
sum :=  0
oldH[] :=  h
oldA[] :=  a

 //init the hubs, authorities
 static initialized = false
 if not initialized
  for y :=  0 to n-1
   sum :=  0
   for x :=  0 to n-1
     sum := sum + g[y*n + x]
   h[y] := sum

  for y :=  0 to n-1
   sum :=  0
   for x :=  0 to n-1
     sum :=  sum + g[x*n + y]
   a[y] :=  sum

 initialized :=  true

 else
  //Calculate Designated A
  for y in first to last
   sum :=  0

   for x in 0 to n - 1
     sum :=  sum + g[x*n+y]*oldH[x]
   a[y] :=  sum

  //calculate Designated H
  for y in first to last
    sum :=  0
    for x in 0 to n - 1
     sum :=  sum + g[y*n+x]*oldA[x]
    h[y] :=  sum

 return h,a
\end{verbatim}
%\end{lstlisting}

\Section{Development Environment}
 % \addcontentsline{toc}{Section}{Development Environment}
This project utilized the C++ programming language and the standard
template library that is not present in the C language.  The initial
development of the serial algorithm occurred on stand-alone PCs, and
then later ported over to MCSR Sweetgum \cite{sweetgum} and MCSR
Mimosa \cite{mimosa} for comparisons of the serial and parallel
implementation. The testing of the parallel implementation occurred
only on Sweetgum and Mimosa. All results were compared with each
other to verify the results between the serial/parallel
implementation and the implementation across platforms.

%\subSection{MCSR Mimosa}
% % \addcontentsline{toc}{subSection}{MCSR Mimosa}
%
%Mimosa is a Linux (Beowulf) cluster. It is a hodge-podge of various
%types of desktop systems. MCSR\cite{mimosa} lists them as follows:
%
%  \newcounter{Lcount1}
%  \begin{list}{\Roman{Lcount1}}
%    {\usecounter{Lcount1}
%    \setlength{\rightmargin}{\leftmargin}}
% \item (12) Compaq DeskPro EP PIII 500: 10 GB IDE Hard Drive: 576 MB ECC SDRAM DIMM Intel PRO/1000 MT Gigabit PCI NIC card.
% \item (40) Dell Optiplex GX400: Pentium 4 1.4 GHz, 40 GB IDE Disk drive, 512 MB RDRAM (400MHz Bus), Intel PRO/1000 MT Gigabit PCI NIC card
% \item (26) Dell Optiplex GX400: Pentium 4 1.8 GHz, 20 GB IDE Disk drive, 512 MB RDRAM (400MHz Bus), Intel PRO/1000 MT Gigabit PCI NIC card
% \item (32) Dell Optiplex GX800: Pentium 4 2.8 GHz, 80 GB IDE Disk drive, 1024 MB SDRAM (800MHz Bus), Intel PRO/1000 MT Gigabit PCI NIC card
% \item (66) CSI Pentium 4 2.8 GHz, 80 GB IDE Disk Drive, 1024 MB RDRAM (533MHz Bus), Intel PRO/1000 MT Gigabit PCI NIC card
% \item (74) Gateway E-6000 Pentium 4 2.8 GHz, 80 GB IDE Disk Drive, 1024 MB RDRAM (400MHz Bus), Intel PRO/1000 MT Gigabit PCI NIC card
% \item Also: Network Attached Storage (shared) with 8 x 250 GB IDE Disk Drives, 3.2 GHz and 1024 MB RDRAM, APC 42U closed cabinets, 6 Nortel BayStack Model 5510-48T NETWORK SWITCHES, 48 10/100/1000Base-T ports with 2 built-in fiber mini-GBIC slots
% \end{list}

%\subSection{MCSR Sweetgum}
% % \addcontentsline{toc}{subSection}{MCSR Sweetgum}
Sweetgum is an SGI Origin 2800, with 128 CPUs. 64 processors are
300MHz MIPS, and 64 are 195 MHz, with 64-bit words. As of this
writing, Sweetgum runs IRIX 6.5.17, a variation of UNIX produced by
SGI. The library used for message passing on Sweetgum was MPI-CH
1.2.1. \cite{sweetgum}. MCSR Mimosa, a hybrid Linux cluster over
Ethernet, was also used for comparison on small data sets
\cite{mimosa}.

\Section{Analysis}
 % \addcontentsline{toc}{Section}{Analysis}
\begin{table}[h]
\center
\begin{tabular}{|l|l|}
  \hline
  Calculate Hubs and Authorities & $2n^{2}$ \\
  Normalize Hubs, Authorities & $2n$ \\
  Repeat i times & $i*(prev\ steps)$ \\
  \hline
\end{tabular}
  \caption{Analysis for serial HITS algorithm}\label{Analysis for serial HITS algorithm}
\end{table}

\begin{table}[h]
\center
\begin{tabular}{|l|l|}
  \hline
  1 Broadcast the adjacency matrix & $n^{2}t_c$\\
  2 Calculate Hubs and Authorities & $2\frac{n^{2}}{p}$ \\
  3 Normalize Hubs, Authorities & $2n$ \\
  4 Gather Hubs, Authorities & $2nt_c$\\
  5 Repeat from 2 i times & $i*(prev\ steps)$ \\
  \hline
\end{tabular}
  \caption{Analysis for general parallel HITS algorithm}\label{Analysis for general parallel HITS algorithm}
\end{table}

\begin{equation} \label{Serial Time} T_{s} = 2i(n^{2}+n) = \Theta(in^{2}) \end{equation}
\begin{equation} \label{Parallel Time} T_{p} = n^{2}t_c + 2i(\frac{n^{2}}{p}+nt_c+n) =
\Theta((i+1)\frac{n^{2}}{p}) \approx
\Theta(i\frac{n^{2}}{p})\end{equation} Loading the file and sorting
the final results are inherently serial processes, which occur only
at process 0. They are not included in our analysis, because they
are common to both the serial and parallel implementations. If the
first term (initial broadcast / start-up cost) is ignored, speedup S
and efficiency E are approximated by

\begin{equation}\label{speedup} S = \frac{T_s}{T_p} = \frac{(1 + n) p}{n + p + pt_c}\end{equation}.
\begin{equation}\label{efficiency}E = \frac{S}{p} = \frac{(1 + n)}{n + p + pt_c} \approx \frac{n}{n + p(1 + t_c)} = \frac{1}{1 + (1 + t_c)\frac{p}{n}} \end{equation}.

This is scalable for large n, since $Limit\ n\ \rightarrow \infty\ S
= p$. The definitions of scalability, efficiency, and speedup are
provided by Grama, et al \cite{grama}. In equation \ref{speedup}, we
can manipulate $p$ for any given $n$ so that $\frac{p}{n}$ remains
constant. This analysis does not account for architectural
limitations. Note also that if the number of iterations is small,
the start-up cost is much more significant in the analysis.
Kleinberg \cite{kleinberg} places the recommended number of
iterations before convergence between 20 and 100 for large sets of
pages, so we do not consider the start-up term.
\begin{table}[h]
\center
\begin{tabular}{|l|l|}
  \hline
  1 Broadcast the Adj Matrix & $ (t_{s}+t_{w}bn^{2})log(p)$ \\
  2 Calculate H, A values & $2\frac{n^{2}}{p}$ \\
  3 Normalize H, A values & $2n$ \\
  4 Gather H, A & $2n\log(p)(t_{s}+t_{w}n)$ \\
  5 Repeat from 2, i times & $i*(previous\ steps)$ \\
  \hline
\end{tabular}
  \caption{Analysis for parallel algorithm on a generalized architecture}\label{parallel analysis}
\end{table}

\begin{equation}\label{Parallel Time } T_{p} = 2i(\frac{n^{2}}{p}+2t_{s}log(p)+t_{w}dn(p-1+log(p)+n)\end{equation}\\

The speedup associated with parallelization of the HITS algorithm
comes from reducing the number of operations involved in the
calculation of hubs and authority values. Assuming that each process
is delegated the same number of vertexes, each process calculates
its share of $2n/p$ values in parallel. In order for the algorithm
to be efficient, we find the relationship between the number of
processes $p$ and the number of vertices $n$ which minimizes
$T_{p}$. Let $t_{w}$ be the time to transfer one item of data,
$t_{s}$ be the start-up time, $p$ be the total number of processors
and $n$ be the total number of vertices. Solving this equation
\ref{Minimal Parallel Time} for p gives the optimal number of
processes for graph with $n$ vertices.
\begin{equation} \label{Minimal Parallel Time} \frac{dT_{p}}{dp} = 2i(\frac{n^{2}}{p^{2}} + 2t_{s}\frac{1}{p} + nt_{w}(1+\frac{1}{p}) = 0\end{equation}

Ignoring the leading constant, and solving for $p (n)$, we obtain
\ref{Minimal Parallel Time 2}. Choosing $p$ processors as a function
of the number of pages $n$ to rank will optimize the total parallel
run time (wall time) of the ranking. Note that no matter what the
input size is, given an infinite parallel system, we can choose some
number of processors that gives an optimal run time.

\begin{eqnarray} \label{Minimal Parallel Time 2} \rightarrow \frac{n^{2}}{p^{2}} + nt_{w} + \frac{1}{p}\ (2t_{s} + nt_{w} ) = 0 \\ \rightarrow p(n) = \lceil{\frac{-2t_s-nt_w+\sqrt{-4n^3t_w+{(2t_s+nt_w)}^3}}{2nw}}\rceil\end{eqnarray}

%%\begin{equation} \Rightarrow p = \frac{-\left( 2\,a t_{s} +
%%      a\,n\,t_{w} +
%%      {\sqrt{4\,n^3\,t_{w} +
%%          {\left( -2\,a\,t_{s} -
%%              a\,n\,t_{w}\right) }^2}}
%%      \right) }{2\,n t_{w}}
%%\end{equation}
%%
%%This is taken directly from the quadratic formula, and given perfect
%%information should give a reasonable estimate on the number of
%%processors which should be used to minimize parallel run time. The
%%parallel overhead is calculated $T_{o} = pT_{p} - T_{s}$\..

%
%\begin{table}[h]
%\center
%\begin{tabular}{|l|l|}
%  \hline
%  % after \\: \hline or \cline{col1-col2} \cline{col3-col4} ...
%  Broadcast the Adjacency Matrix & $ (t_{s}+t_{w}bn^{2})log(p)$ \\
%  Calculate Hubs and Authorities & $2\frac{n^{2}}{p}$ \\
%  Gather Hubs and Authorities & $2(t_{s}log(p)+t_{w}dn(p-1))$ \\
%  Normalize Hubs, Authorities & $2n$ \\
%  Broadcast Hubs, Authorities & $2log(p)n(t_{s}+t_{w}dn)$ \\
%  Repeat i times & $i*(previous steps)$ \\
%  \hline
%\end{tabular}
%  \caption{Analysis for simple parallel algorithm on a hypercube architecture}\label{Analysis for simple parallel algorithm}
%\end{table}
%
%\begin{equation}  \label{Parallel Time (neglecting initial Broadcast)} T_{p} = 2i(\frac{n^{2}}{p}+2t_{s}log(p)+t_{w}dn(p-1+log(p)+n) = \Theta(n^{2}) \end{equation}\\
%
%Clearly, the Speedup associated with parallelization of the HITS algorithm comes from reducing the number of operations involved in the calculation of hubs an authority values. Assuming that each process is delegated the same number of vertexes, each process calculates its share of $2n/p$ values in parallel. In order for the algorithm to be efficient, we find the relationship between the number of processes $p$ and the number of vertices $n$ which minimizes $T_{p}$. Let $t_{w}$ be the time to transfer one word of data, $t_{s}$ be the startup time, $d$ be the number of words used by a double, $d$ be the number of words used by a boolean value (unpacked), $a=1/Log(2)$, $p$ be the total number of processors and $n$ be the total number of vertices. Using differential calculus we have
%
%\begin{equation} \label{Minimal Parallel Time} \frac{dT_{p}}{dp} = 2i(\frac{n^{2}}{p^{2}} + 2t_{s}\frac{1}{p*log(2)} + t_{w}dn(1+\frac{1}{p*log(2)}) = 0\end{equation}
%
%\begin{equation} \Rightarrow p = \frac{-\left( 2\,a t_{s} +
%      a\,d\,n\,t_{w} +
%      {\sqrt{4\,d\,n^3\,t_{w} +
%          {\left( -2\,a\,t_{s} -
%              a\,d\,n\,t_{w}\right) }^2}}
%      \right) }{2\,d n t_{w}}
%\end{equation}
%
%This is taken directly from the quadratic formula, and given perfect
%information should give a reasonable estimate on the number of
%processors which should be used to minimize parallel run time. The
%parallel overhead is calculated $T_{o} = pT_{p} - T_{s}$\..
%
%\begin{equation} \label{Parallel Overhead)} T_{O} = 2i(2t_{s}plog(p)+t_{w}dnp(p-1+log(p))+np - n) \end{equation}
%
%Similarly, speedup is given by $S = \frac{T_{s}}{T_{p}} $, and
%efficiency is $E = \frac{T_{s}}{pT_{p}} $. Notice the algorithm is
%cost optimal and scalable.
%
%\begin{equation} \label{Speedup)} S = \frac{n^{2}+n}{\frac{n^{2}}{p}+2t_{s}log(p)+t_{w}dn(p-1+log(p)+n)} = \Theta(1) \end{equation}
%\begin{equation} \label{Efficiency)} E = \frac{n^{2}+n}{(n^{2}+2pt_{s}log(p)+t_{w}pdn(p-1+log(p)+np)} = \Theta(1) \end{equation}

\Section{Results}
 % \addcontentsline{toc}{Section}{Results}
\begin{figure}[h]
  \begin{center}
    \includegraphics[scale=0.3]{images/sweetgum_usm1.png}
    \caption{\label{time_usm} Wall time vs processors on sweetgum w/ real-world input size of n=13505 over five runs. }
  \end{center}
\end{figure}
\begin{figure}[h]
  \begin{center}
    \includegraphics[scale=0.3]{images/sweetgum_sp_usm1.png}
    \caption{\label{speedup_usm} Speedup vs processors on sweetgum w/ real-world input size of n=13505 over five runs. }
  \end{center}
\end{figure}
\begin{figure}[h]
  \begin{center}
    \includegraphics[scale=0.3]{images/sweetgum_ef_usm1.png}
    \caption{\label{efficiency_usm} Efficiency vs processors on sweetgum w/ real-world input size of n=13505 over five runs. }
  \end{center}
\end{figure}
We used a web spider to create an adjacency list for each of
approximately 13500 interlinked web pages in the University of
Southern Mississippi (usm.edu) domain. An adjacency list (or matrix)
is the only input for our algorithm, as requires solely topological
features as input. The total run time for our data set of 13500
nodes was a negative exponential function of the number of
processors. Speedup was expectedly logarithmic.

For figure \ref{time_usm}, the curve fit is $807.65 + -27.68x +
4225.78e^{-x}$, and is averaged over five data sets for 1 to 24
processes. The average serial time over this data set was 2350.32
seconds. The curve fit shown in figure \ref{speedup_usm} is $2.34729
Lg(x)$, where Lg is the natural logarithm, as expected by our
analysis. Note that in Figure \ref{speedup_usm}, the graph is
monotonically increasing, which is a property of a scalable
algorithm. \Section{Conclusion}
 % \addcontentsline{toc}{Section}{Conclusion}

In this paper, we have presented a parallel implementation of the
HITS page-ranking algorithm \cite{kleinberg}. We show graphically
that a naive block-and-broadcast, centralized implementation is not
scalable. We then improved the algorithm using the all-to-all gather
communications pattern.

Analysis showed that the algorithm is scalable (for many
iterations), and the results obtained for a large network of
constant size support our analysis. Future work will provide a
three-variable numerical result graph of speedup vs number of
vertices and number of processes, showing iso-efficiency lines. The
algorithm can also be adapted to massively parallel networks,
perhaps three orders of magnitude larger than the medium sized page
network used to obtain results. Collecting this data itself in an
ethical manner is time consuming, as many web sites prefer to
spiders to not index them. Some performance tweaks may also be
required to make the algorithm run in near-asymptotic time. Since
Google Page Rank \cite{google} works in a similar manner to the HITS
algorithm, parallelization may be appropriate. This algorithm is not
context-free like HITS\cite{kleinberg}, which may yield special
challenges.

\Section{Acknowledgement}
 % \addcontentsline{toc}{Section}{Acknowledgement}
We would like to thank the Mississippi Center for Supercomputing
Research at the University of Mississippi for providing the
computing resources used in obtaining our results \cite{sweetgum},
\cite{mimosa}, and for providing support services.
\begin{thebibliography}{8}
 % \addcontentsline{toc}{Section}{References}

\bibitem{kleinberg} Kleinberg, J. {\bf Authoritative Sources in a Hyperlinked Environment.} Journal~of~the~ACM, Vol. 46, No. 5, September 1999, pp. 604�--632.

\bibitem{gulli} Gulli, A. Signorini, A. {\bf The indexable web is more than 11.5
billion pages.} International World Wide Web Conference. New York:
ACM. January 2005. ISBN 1595930515.

\bibitem{corman} Cormen T. Leiserson C. Rivest R. Stein, S. {\bf Introduction to Algorithms}, Second Edition. MIT
Press and McGraw-Hill, 2001. ISBN 0262032937. Section 22.1:
Representations of graphs, pp. 527�531.

\bibitem{gropp} Gropp, W. Lusk, E. Skjellum, A. {\bf Using MPI: Portable Parallel Programming with the Message Passing Interface}.
2001. Cambridge: MIT. Scientific Engineering and Computation Series. ISBN 0262571323. Chapter 6: Advanced Message Passing in MPI, pp 121-131.

\bibitem{grama} Grama, A. Gupta, A. Karypis, G. Kumar, V. {\bf Introduction to Parallel
Computing}. 2nd Edition. January 16, 2003. Essex: Addison Wesley.
ISBN 0201648652.

\bibitem{sweetgum} MCSR. {\bf SGI Origin 2800 at  \href{sweetgum.mcsr.olemiss.edu}{sweetgum.mcsr.olemiss.edu} } Acessed 12-05-05 from \href{http://www.mcsr.olemiss.edu/computing/sweetgum.html}{http://www.mcsr.olemiss.edu/computing/sweetgum.html}

\bibitem{mimosa} MCSR. {\bf Beowulf Cluster at  \href{mimosa.mcsr.olemiss.edu}{mimosa.mcsr.olemiss.edu} } Acessed 12-05-05 from \href{http://www.mcsr.olemiss.edu/computing/mimosa2.html}{http://www.mcsr.olemiss.edu/computing/mimosa2.html}

\bibitem{google} Google Corporation. {\bf Google searches more sites more quickly, delivering the most relevant
results.} Acessed 05-18-06 from
\href{http://www.google.com/technology/}{http://www.google.com/technology}

\end{thebibliography}

%\include{index}
% % \addcontentsline{toc}{Section}{Index}
\end{document}
